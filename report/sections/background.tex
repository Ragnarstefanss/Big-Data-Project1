DHT is decentralized lookup service with a similar interface to normal hash tables. It stores key-value pairs across multiple different nodes. Nodes can be added or removed from DHT with minimal redistribution of keys. As with a normal hash table, keys are mapped to any value and serve as a unique identifier for that value. Each node can either retrieve the value or forward the query to the appropriate node.\cite{DHT-History}

Chord is an overlay network for DHT to efficiently handle routing between nodes. Each node's IP address is hashed (using the consistent hashing algorithm SHA-1) to some id and placed in an identifier circle modulo $2^m$ and maintains links to the previous node and the node responsible for the id with offsets $2^0, 2^1, \ldots, 2^{m-1}$ from itself where $m$ must be chosen such that probability of collision is minuscule. Note that the offset is increasing exponentially so each node knows more about nodes closer than those further away. The latter links are called finger tables and it is what Chord uses to reach the desired destination efficiently. When asking for a value given a key, that key is also hashed and the next node id succeeding that value (wrapping around the circle) is the one responsible for it.\cite{chord-peer-to-peer}

Suppose there are two nodes with ids $551$ and $813$ and no other nodes have ids in $[551,813]$. Keys can share ids with nodes and a key mapped to $551$ belongs to the node with the same id while any key mapped to $(551,813]$ belongs to the node with id $813$.

When nodes join and leave, Chord makes sure that all nodes in a Chord circle are connected with a successor and predecessors. When multiple nodes are joining concurrently it becomes very difficult to make sure that all finger tables are still correct. It uses a background task called Stabilization to periodically update finger tables to ensure correct look ups and notify nodes about joining and leaving nodes.\cite{chord-peer-to-peer}
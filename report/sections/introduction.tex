The initial research for Distributed hash table (DHT) was motivated by rising popularity in systems using peer-to-peer (P2P). These systems provided a single interface for file sharing that utilized resources distributed over the internet made possible with increasing network bandwidth and disk space. All of these P2P systems used their own ways to locate the data provided by their peers but the benefits of using DHT was that it had a more reliable structure of key based routing. They were decentralized and gave efficient and correct results while still being reliable given the unreliability of nodes (servers).\cite{DHT-History} There are multiple different DHT infrastructures in existence. One of these is Chord \cite{chord-peer-to-peer} that we base our project on.

The rest of this paper is structured as follows. Chapter \ref{ch:backgr} introduces key terminology and concepts. In chapter \ref{ch:methods} we describe our implementation. In chapter \ref{ch:res} we specify the experimental setup and finally interpret the results in Chapter \ref{ch:conclusion}.

